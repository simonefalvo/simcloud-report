\section{Modello di Specifica}
In questo modello vengono definite le variabili e le equazioni necessarie al
calcolo delle metriche, inoltre viene descritta la logica con cui il sistema
evolve a seguito degli eventi che si verificano ed i vincoli a cui esso deve
sottostare.
%
%
\subsection{Variabili}
Di principale interesse sono le variabili che compongono lo stato del sistema,
ovvero quelle che lo caratterizzano completamente, esse tengonono conto del
numero di job in servizio, suddivisi per classe e per nodo di esecuzione, ad
ogni istante di tempo e sono indicate nella tabella~\ref{state}.
%
\begin{table}[!h]
\begin{tabular}{c|l}
{$n_1^{clet}(t)$}  & numero di job di classe 1 nel cloudlet al tempo $t$\\
{$n_2^{clet}(t)$}  & numero di job di classe 2 nel cloudlet al tempo $t$\\
{$n_1^{cloud}(t)$} & numero di job di classe 1 nel cloud al tempo $t$\\
{$n_2^{cloud}(t)$} & numero di job di classe 2 nel cloud al tempo $t$\\
{$n_{setup}(t)$}   & numero di job in fase di setup al tempo $t$\\
\end{tabular}
\centering
\caption{Variabili che compongono lo stato del sistema}
\label{state}
\end{table}
%

Le altre variabili utili al calcolo delle statistiche del sistema sono indicate
nella tabella~\ref{vars} e riguardano i tempi di risposta ed i completamenti,
sempre suddivise per classe e nodo di esecuzione.
%
\begin{table}[!h]
\begin{tabular}{c|l}
{$s_{1,i}^{clet}$}     & tempo di servizio dell’i-esimo job di classe 1 eseguito
nel cloudlet \\[2pt]
{$s_{2,i}^{clet}$}     & tempo di servizio dell’i-esimo job di classe 2 eseguito
nel cloudlet \\[2pt]
{$s_{1,i}^{cloud}$}    & tempo di servizio dell’i-esimo job di classe 1 eseguito
nel cloud \\[2pt]
{$s_2^{cloud,i}$}      & tempo di servizio dell’i-esimo job di classe 2 eseguito
nel cloud \\[2pt]
{$s_{intr,i}^{clet}$}  & tempo di servizio nel cloudlet dell’i-esimo job
interrotto \\[2pt]
{$s_{intr,i}^{cloud}$} & tempo di servizio nel cloud dell’i-esimo job interrotto
\\[2pt]
{$s_i^{setup}$}        & tempo di setup dell’i-esimo job interrotto \\[2pt]
{$c_1^{clet}(t)$}      & numero di job di classe 1 completati nel cloudlet al
tempo $t$ \\[2pt]
{$c_2^{clet}(t)$}      & numero di job di classe 2 completati nel cloudlet al
tempo $t$ \\[2pt] 
{$c_1^{cloud}(t)$}     & numero di job di classe 1 completati nel cloud al tempo
$t$ \\[2pt] 
{$c_2^{cloud}(t)$}     & numero di job di classe 2 completati nel cloud al tempo
$t$ \\[2pt]
{$n_{intr}(t)$}        & numero di job interrotti al tempo $t$ \\
\end{tabular}
\centering
\caption{Variabili tempi di servizio e completamenti}
\label{vars}
\end{table}
%
%
\subsection{Vincoli}
Una volta definito lo stato del sistema e le variabili che lo compongono è
necessario definire anche come esse sono relazionate ed i vincoli a cui sono
sottoposte.
\begin{enumerate}
\item Ad ogni istante di tempo $t$ devono valere le seguenti condizioni:
\begin{eqnarray*}
n_1^{clet}(t) + n_2^{clet}(t) & \leq N & \\
n_1^{clet}(t) + n_2^{clet}(t) & \leq S & \qquad \textrm{se } n_2^{clet}(t) > 0
\end{eqnarray*}
\item Stato iniziale ($t=t_{start}$):
\begin{eqnarray*}
n_1^{clet}(t) = n_2^{clet}(t) = n_1^{cloud}(t) = n_2^{cloud}(t) = n_{setup}(t) &=& 0 
\\
c_1^{clet}(t) = c_2^{clet}(t) = c_1^{cloud}(t) = c_2^{cloud}(t) = c_{setup}(t) &=& 0
\end{eqnarray*}
\item Il primo evento deve essere un arrivo
\item Dopo l’arrivo di un numero prefissato di job, i successivi arrivi vengono
ignorati e non più processati.  Il processo di arrivo si interrompe quando viene
processato un numero prefissato di job.  L’ultimo evento corrisponde al
completamento dell’ultimo job.
\item La selezione dei server all’interno di un nodo non è regolata da nessun
algoritmo, poiché metriche relative ai singoli server non sono rilevanti ai fini
dell’applicazione.  
\end{enumerate}
%
%
\subsection{Metriche}
\subsubsection{Tempi di risposta}
Poiché i nodi del sistema sono sprovvisti di code, i tempi medi di risposta
corrispono ai tempi medi di servizio, il calcolo viene quindi ridotto al
rapporto tra la somma dei tempi di servizio sperimentati dai job in esame e la
loro quantità, pertanto, per prima cosa è utile calcolare le somme dei
tempi:
\begin{displaymath}
s_j^{clet} = \displaystyle \sum_{i=1}^{c_j^{clet}(t_{stop})} s_{j,i}^{clet} 
\qquad\quad
s_j^{cloud} = \displaystyle \sum_{i=1}^{c_j^{cloud}(t_{stop})} s_{j,i}^{cloud}
\qquad \quad j = 1, 2 
%s_{intr}^{clet} = \sum_{i=1}^{n_{intr}(t_{stop})} s_{intr,i}^{clet} &\quad&
%s_{intr}^{cloud} = \sum_{i=1}^{n_{intr}(t_{stop})} s_{intr,i}^{cloud} \ \qquad
%\ s_{setup} = \sum_{i=1}^{n_{intr}(t_{stop})} s_i^{setup} 
\end{displaymath}
\begin{displaymath}
s_{intr} = \sum_{i=1}^{n_{intr}(t_{stop})} (s_{intr,i}^{clet} +
s_{intr,i}^{cloud} + s_i^{setup})
\end{displaymath}
%
Con queste formule si è ottenuta la somma dei tempi di servizio dei job
suddivisi per classe e nodo di esecuzione, ed anche una relativa esclusivamente
ai job interrotti in cui vengono sommati i tempi associati ai differenti nodi
che percorrono (cloudlet, setup e cloud).

È importante notare che il numero di job coinvolti è relativo ad un'istante di
tempo corrispondente alla fine della finestra di osservazione $[t_{start};
t_{stop}]$, ove si è posto per semplicità $t_{start}=0$.

Adesso per il calcolo dei tempi di risposta rimane da fare il rapporto con i
completamenti relativi ai nodi oppure alle classi di interesse.
%
\setlength\arraycolsep{2pt}
\begin{eqnarray}
\label{eq:sjclet}
E[T_j^{clet}] = E[S_j^{clet}] &=& \frac{s_j^{clet}}{c_j^{clet}(t_{stop})}
\ \qquad \ \qquad \ j = 1, 2 
\\[10pt]
\label{eq:sjcloud}
E[T_j^{cloud}] = E[S_j^{cloud}] &=&
\frac{s_j^{cloud}}{c_j^{cloud}(t_{stop})}
\qquad \ \qquad j = 1, 2 
\\[10pt]
\label{eq:sintr}
E[T_{intr}] = E[S_{intr}] & = &
\frac{s_{intr}}{n_{intr}(t_{stop})}
\end{eqnarray}
\begin{eqnarray}
\label{eq:s1}
E[T_1] = E[S_1] & = &
\frac{s_1^{clet} + s_1^{cloud}}{c_1^{clet}(t_{stop}) + c_1^{cloud}(t_{stop})}
\\[10pt]
\label{eq:s2}
E[T_2] = E[S_2] & = &
\frac{s_1^{clet} + s_1^{cloud} + s_{intr}}{c_2^{clet}(t_{stop}) +
c_2^{cloud}(t_{stop})} \\[10pt]
\label{eq:s}
E[T] = E[S] & = &
\frac{s_1^{clet} + s_1^{cloud} + s_2^{clet} + s_2^{cloud} + s_{intr}}
{c_1^{clet}(t_{stop}) + c_1^{cloud}(t_{stop}) + c_2^{clet}(t_{stop}) +
c_2^{cloud}(t_{stop})} 
\end{eqnarray}
%
Nella formula~\ref{eq:sintr} la somma dei tempi di servizio dei job interrotti
viene divisa per il numero di interruzioni, che equivale al numero di
completamenti.\\
Nella formula~\ref{eq:s2}, in cui si calcola il tempo di risposta dei job di
classe 2, vengono considerati i job della suddetta classe che passano
esclusivamente per il cloudlet e per il cloud oltre ai job che subiscono le
interruzioni, ma al denominatore non è presente il numero di job interrotti
$n_{intr}$ perché è già incluso nella variabile $c_2^{cloud}$, essendo tali job
completati nel cloud. Lo stesso discorso vale per la formula~\ref{eq:s}.
%
\subsubsection{Popolazione media}
La popolazione media viene calcolata integrando le variabili in un'intervallo di 
tempo corrispondente alla finestra di osservazione della simulazione e dividendo
per la lunghezza di quest'ultima. Le metriche globali possono essere calcolate
sommando le opportune metriche locali, poiché sono relative tutte allo stesso
intervallo di tempo.
\begin{eqnarray}
\label{eq:njclet}
E[N_j^{clet}] & = &
\frac{1}{t_{stop} - t_{start}} 
\displaystyle \int_{t_{start}}^{t_{stop}} n_j^{clet}(t) \ dt
\qquad j = 1, 2 
\\[10pt]
\label{eq:njcloud}
E[N_j^{cloud}] & = &
\frac{1}{t_{stop} - t_{start}} 
\displaystyle \int_{t_{start}}^{t_{stop}} n_j^{cloud}(t) \ dt
\qquad j = 1, 2 
\\[10pt]
\label{eq:nsetup}
E[N_{setup}] & = &
\frac{1}{t_{stop} - t_{start}} 
\displaystyle \int_{t_{start}}^{t_{stop}} n_{setup}(t) \ dt
\\[10pt]
\label{eq:n1}
E[N_{1}] & = & E[N_1^{clet}] + E[N_1^{cloud}]
\\[10pt]
\label{eq:n2}
E[N_{2}] & = & E[N_2^{clet}] + E[N_2^{cloud}] + E[N_{setup}]
\\[10pt]
\label{eq:nclet}
E[N_{clet}] & = & E[N_1^{clet}] + E[N_2^{clet}]
\\[10pt]
\label{eq:ncloud}
E[N_{cloud}] & = & E[N_1^{cloud}] + E[N_2^{cloud}]
\\[10pt]
\label{eq:n}
E[N] & = & E[N_{cloud}] + E[N_{clet}] + E[N_{setup}] 
\\   & = & E[N_{1}] + E[N_{2}]
\end{eqnarray}
%
Può essere utile ricordare che i job interrotti sono job di classe 2, pertanto
nel calcolo della popolazione media dei job di questa classe nel sistema
(equazione~\ref{eq:n2}), vanno considerati anche i job nel centro di setup.
%
\subsubsection{Throughput}
Il throughput viene calcolato come il numero di job completati in un intervallo
di tempo corrispondente alla finestra di osservazione della simulazione.\\
Come per la popolazione media è sufficiente comporre i vari throughput locali
per ottenere i throughput del sistema, dei singoli nodi oppure delle singole
classi. Per esempio per il calcolo del throghput del sistema
(equazione~\ref{eq:x}) si possono sommare i throughput delle classi oppure i
throughput dei nodi.
\begin{eqnarray}
\label{eq:xjclet}
X_j^{clet} & = & \frac{c_j^{clet}(t_{stop})}{t_{stop} - t_{start}} 
\qquad j = 1, 2 
\\[10pt]
\label{eq:xjcloud}
X_j^{cloud} & = & \frac{c_j^{cloud}(t_{stop})}{t_{stop} - t_{start}} 
\qquad j = 1, 2 
\\[10pt]
\label{eq:xj}
X_j & = & X_j^{clet} + X_j^{cloud}
\qquad j = 1, 2 
\\[10pt]
\label{eq:xclet}
X_{clet} & = & X_1^{clet} + X_2^{clet}
\\[10pt]
\label{eq:xcloud}
X_{cloud} & = & X_1^{cloud} + X_2^{cloud}
\\[10pt]
\label{eq:x}
X & = & X_1 + X_2 = X_{clet} + X_{cloud}
\end{eqnarray}
%%
Il throughput del centro di setup non è particolarmente interessante perché è un
nodo interno che non emette job all'esterno del sistema, funge solo da nodo
intermedio tra cloudlet e cloud, in definitiva il suo throughput non
contriubuisce a quello del sistema.
\subsubsection{Interruzioni}
La percentuale di job interrotti viene calcolata sia rispetto al numero di job di
classe 2 del sistema, sia rispetto al numero di job di classe 2 che passano per
il cloudlet.
\begin{eqnarray}
P_{intr} &=& 
\frac{n_{intr}(t_{stop})}{c_2^{clet}(t_{stop}) + c_2^{cloud}(t_{stop})} \\[10pt]
P_{intr}^{clet} &=& 
\frac{n_{intr}(t_{stop})}{n_{intr}(t_{stop}) + c_2^{clet}(t_{stop})}
\end{eqnarray}
%
\subsection{Eventi}
Lo stato del sistema evolve a seguito di eventi di vario tipo:
\begin{enumerate}
\item{Arrivo di un job di classe 1}
\item{Arrivo di un job di classe 2}
\item{Partenza di un job}
\item{Setup}
\end{enumerate}
%
L'algoritmo~\ref{alg} mostra tale evoluzione e le azioni che vengono intraprese
in ogni possibile caso.\\
In generale ad ogni arrivo viene stabilito il nodo di esecuzione in base allo
stato del cloudlet e vengono aggiornate opportunamente le variabili, nel caso si
debba sostituire un job di classe 2 con uno di classe 1 appena arrivato si
provvede a rimuovere il relativo tempo di servizio $s_2^{clet,k}$
precedentemente aggiunto e a registrare l'ammontare di tempo $s_{intr,k}$ per
cui il job è stato in esecuzione prima che venisse interrotto.  Nel caso in cui
si verifica un evento di partenza viene incrementato la corrispondente variabile
di completamento e nel caso di un evento di setup, il relativo job
precedentemente interrotto può essere finalmente eseguito sul cloud. Si noti
anche che, nell'aggiornamento delle variabili temporali, l'istante $t'$ è
corrisponde al momento in cui si verifica l'evento successivo a quello corrente
che avviene all'istante $t$.
%
\begin{algorithm}[!h]
\centering
\caption{Logica del sistema in base agli eventi}
\label{alg}
\begin{algorithmic}
\STATE \textbf{Arrivo di un job $i$ di classe 1}:
\IF{$n_1^{clet}(t) = N$}
\STATE \emph{esecuzione su cloud}
\STATE $s_1^{cloud} \leftarrow s_1^{cloud} + s_{1,i}^{cloud}$
\STATE $n_1^{cloud}(t') \leftarrow n_1^{cloud}(t) + 1$
\ELSIF{$n_1^{clet}(t) + n_2^{clet}(t) < S$}
\STATE \emph{esecuzione su cloudlet}
\STATE $s_1^{clet} \leftarrow s_1^{clet} + s_{1,i}^{clet}$
\STATE $n_1^{clet}(t') \leftarrow n_1^{clet}(t) + 1$
\ELSIF{$n_2^{clet}(t) > 0$}
\STATE \emph{interruzione e setup job $k$ di classe 2}
\STATE \emph{esecuzione su cloudlet job $i$ di classe 1}
\STATE $s_1^{clet} \leftarrow s_1^{clet} + s_{1,i}^{clet}$
\STATE $s_2^{clet} \leftarrow s_2^{clet} - s_2^{clet,k}$
\STATE $s_{intr} \leftarrow s_{intr} + s_{intr,k}$
\STATE $s_{setup} \leftarrow s_{setup} + s_{setup,k}$
\STATE $n_{setup}(t') \leftarrow n_{setup}(t) + 1$
\STATE $n_1^{clet}(t') \leftarrow n_1^{clet}(t) + 1$
\STATE $n_2^{clet}(t') \leftarrow n_2^{clet}(t) - 1$
\ELSE
\STATE \emph{esecuzione su cloudlet}
\STATE $s_1^{clet} \leftarrow s_1^{clet} + s_{1,i}^{clet}$
\STATE $n_1^{clet}(t') \leftarrow n_1^{clet}(t) + 1$
\ENDIF
\STATE \textbf{Arrivo di un job $i$ di classe 2}:
\IF{$n_1^{clet}(t) + n_2^{clet}(t) \geq S$}
\STATE \emph{esecuzione su cloud}
\STATE $s_2^{cloud} \leftarrow s_2^{cloud} + s_2^{cloud,i}$
\STATE $n_2^{cloud}(t') \leftarrow n_2^{cloud}(t) + 1$
\ELSE
\STATE \emph{esecuzione su cloudlet}
\STATE $s_2^{clet} \leftarrow s_2^{clet} + s_{2,i}^{clet}$
\STATE $n_2^{clet}(t') \leftarrow n_2^{clet}(t) + 1$
\ENDIF
\STATE \textbf{Partenza di un job di classe $j$ dal cloudlet}:
\STATE $c_j^{clet}(t') \leftarrow c_j^{clet}(t) + 1$
\STATE $n_j^{clet}(t') \leftarrow n_j^{clet}(t) - 1$
\STATE \textbf{Partenza di un job di classe $j$ dal cloud}:
\STATE $c_j^{cloud}(t') \leftarrow c_j^{cloud}(t) + 1$
\STATE $n_j^{cloud}(t') \leftarrow n_j^{cloud}(t) - 1$
\STATE \textbf{Setup}:
\STATE \emph{esecuzione su cloud}
\STATE $s_2^{cloud} \leftarrow s_2^{cloud} + s_2^{cloud,i}$
\STATE $n_{setup}(t') \leftarrow n_{setup}(t) - 1$
\STATE $n_2^{cloud}(t') \leftarrow n_2^{cloud}(t) + 1$
\end{algorithmic}
\end{algorithm}
%
%
