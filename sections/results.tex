\section{Risultati}
La validazione dei modelli appena discussi è stata effettuata confrontando i
risultati della simulazione con le stime dedotte dall'analisi, valutando quattro
possibili scenari contraddistinti dal valore che assume il parametro $S$.

Ognuno di questi scenari è stato simulato facendo processare al programma lo
stesso numero di job affinché le statistiche globali del sistema possano
essere confrontate. Tale numero è stato stabilito in base alle metriche che
andavano valutate ed al numero di job che le riguardava.

Lo scenario con valore di soglia $S=5$, è quello in cui soltanto circa il $2\%$
dei job di classe 2 è processato nel cloudlet e di questi circa il $90\%$
vengono interrotti, ne risulta, quindi che in relazione al numero totale dei job
che transitano nel sistema, solamente lo $0.14\%$ sono job di classe 2
processati con successo nel cloudlet e l'$1.37\%$ sono job interrotti. In
definitiva, al fine di avere una dimensione dei batch soddisfacente per il
calcolo delle relative metriche, è stato scelto un numero totale di job pari a
$500000$. In questo modo si ottiene un numero di job di classe 2 processati con
successo nel cloudlet pari a circa $700$ e un numero di job interrotti pari a
circa $6850$, ciò significa che per un valore $k=64$ corrispondente al numero
di batch, si ottengono batch di dimensione $10$ e $107$ rispettivamente, che
sono sufficienti a determinare intervalli di confidenza, seppur con un ampio
margine di errore.

Quello appena descritto è il peggior scenario possibile, in cui si registra il
minor numero di job per una determinata metrica, gli altri scenari vengono
quindi simulati tutti con un numero totale di job pari a $500000$, e a seconda
del numero di job che verranno processati nei vari nodi e della loro tipologia,
sono risultati intervalli di confidenza più o meno precisi.

Purtroppo è risultato impossibile determinare risultati attendibili per i job
di classe 1 che vengono processati nel cloud, poiché il loro numero è, per ogni
scenario, talmente esiguo, che per avere un batch di dimensione 10 sarebbero
necessari almeno 3 milioni di job in tutto. Tuttavia, con un numero di
$500000$ job totali, si ottengono delle statistiche globali molto attendibili,
infatti, ad esempio per il tempo medio di risposta del sistema, si ottiene 
una dimensione del batch pari a $\lfloor\frac{500000}{64}\rfloor = 7812$.
%
%
\subsection{Scenario 1: $\mathbf{S=N=20}$}
\subsubsection{Tempi di risposta}

\begin{table}[!h]
\begin{tabular}{c|c|c|c|c|c|c|c|c|c|c|c|c}
                & R1 & R2 & R3 & R4 & R5 & R6 & R7 & R8 & R9 & R10 & STIMA & MAX ERR \\
                \hline
$s_1^{clet}$  & R1 & R2 & R3 & R4 & R5 & R6 & R7 & R8 & R9 & R10 & STIMA & MAX ERR \\
$s_2^{clet}$  & R1 & R2 & R3 & R4 & R5 & R6 & R7 & R8 & R9 & R10 & STIMA & MAX ERR \\
$s_{clet}$    & R1 & R2 & R3 & R4 & R5 & R6 & R7 & R8 & R9 & R10 & STIMA & MAX ERR \\
$s_1^{cloud}$ & R1 & R2 & R3 & R4 & R5 & R6 & R7 & R8 & R9 & R10 & STIMA & MAX ERR \\
$s_2^{cloud}$ & R1 & R2 & R3 & R4 & R5 & R6 & R7 & R8 & R9 & R10 & STIMA & MAX ERR \\
$s_{cloud}$   & R1 & R2 & R3 & R4 & R5 & R6 & R7 & R8 & R9 & R10 & STIMA & MAX ERR \\
$s_{setup}$   & R1 & R2 & R3 & R4 & R5 & R6 & R7 & R8 & R9 & R10 & STIMA & MAX ERR \\
$s_{intr}$    & R1 & R2 & R3 & R4 & R5 & R6 & R7 & R8 & R9 & R10 & STIMA & MAX ERR \\
$s_1$         & R1 & R2 & R3 & R4 & R5 & R6 & R7 & R8 & R9 & R10 & STIMA & MAX ERR \\
$s_2$         & R1 & R2 & R3 & R4 & R5 & R6 & R7 & R8 & R9 & R10 & STIMA & MAX ERR \\
$s$           & R1 & R2 & R3 & R4 & R5 & R6 & R7 & R8 & R9 & R10 & STIMA & MAX ERR \\
\end{tabular}
\centering
\caption{Confronto simulazione-stima per $S=20$}
\label{srv_20}
\end{table}

\begin{table}[!h]
\begin{adjustbox}{width=\textwidth}
\begin{tabular}{c|c|c|c|c|c|c}
                & R1 & R2 & R3 & R4 & STIMA & MAX ERR \\
                \hline
$n_1^{clet}$  &$8.9260\pm0.0050$& $8.8170\pm0.0130$ & $8.9151\pm0.0082$ &
$8.8811\pm0.0075$ & & \\
$n_2^{clet}$  & R1 & R2 & R3 & R4 & STIMA & MAX ERR \\
$n_{clet}$    & R1 & R2 & R3 & R4 & STIMA & MAX ERR \\
$n_1^{cloud}$ & R1 & R2 & R3 & R4 & STIMA & MAX ERR \\
$n_2^{cloud}$ & R1 & R2 & R3 & R4 & STIMA & MAX ERR \\
$n_{cloud}$   & R1 & R2 & R3 & R4 & STIMA & MAX ERR \\
$n_{setup}$   & R1 & R2 & R3 & R4 & STIMA & MAX ERR \\
$n_1$         & R1 & R2 & R3 & R4 & STIMA & MAX ERR \\
$n_2$         & R1 & R2 & R3 & R4 & STIMA & MAX ERR \\
$n$           & R1 & R2 & R3 & R4 & STIMA & MAX ERR \\
\end{tabular}
\end{adjustbox}
%\centering
\caption{Risultati popolazione media per $S=20$}
\label{pop_20}
\end{table}

\subsubsection{Popolazione}
\subsubsection{Throughput}
%
%
\subsection{Scenario 2: $\mathbf{S=\frac{N}{2}=10}$}
\subsubsection{Tempi di risposta}
\subsubsection{Popolazione}
\subsubsection{Throughput}
%
%
\subsection{Scenario 3: $\mathbf{S=\frac{3}{4}N=15}$}
\subsubsection{Tempi di risposta}
\subsubsection{Popolazione}
\subsubsection{Throughput}
%
%
\subsection{Scenario 4: $\mathbf{S=\frac{N}{4}=5}$}
\subsubsection{Tempi di risposta}
\subsubsection{Popolazione}
\subsubsection{Throughput}
