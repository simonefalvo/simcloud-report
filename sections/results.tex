\section{Risultati}
La validazione dei modelli appena discussi è stata effettuata confrontando i
risultati della simulazione con le stime dedotte dall'analisi, valutando quattro
possibili scenari contraddistinti dal valore che assume il parametro $S$.

Ognuno di questi scenari è stato simulato facendo processare al programma lo
stesso numero di job affinché le statistiche globali del sistema possano
essere confrontate. Tale numero è stato stabilito in base alle metriche che
andavano valutate ed al numero di job che le riguardava.

Lo scenario con valore di soglia $S=5$, è quello in cui soltanto circa il $2\%$
dei job di classe 2 è processato nel cloudlet e di questi circa il $90\%$
vengono interrotti, ne risulta, quindi che in relazione al numero totale dei job
che transitano nel sistema, solamente lo $0.14\%$ sono job di classe 2
processati con successo nel cloudlet e l'$1.37\%$ sono job interrotti. In
definitiva, al fine di avere una dimensione dei batch soddisfacente per il
calcolo delle relative metriche, è stato scelto un numero totale di job pari a
$500000$. In questo modo si ottiene un numero di job di classe 2 processati con
successo nel cloudlet pari a circa $700$ e un numero di job interrotti pari a
circa $6850$, ciò significa che per un valore $k=64$ corrispondente al numero
di batch, si ottengono batch di dimensione $10$ e $107$ rispettivamente, che
sono sufficienti a determinare intervalli di confidenza, seppur con un ampio
margine di errore.

Quello appena descritto è il peggior scenario possibile, in cui si registra il
minor numero di job per una determinata metrica, gli altri scenari vengono
quindi simulati tutti con un numero totale di job pari a $500000$, e a seconda
del numero di job che verranno processati nei vari nodi e della loro tipologia,
sono risultati intervalli di confidenza più o meno precisi.

Purtroppo è risultato impossibile determinare risultati attendibili per i job
di classe 1 che vengono processati nel cloud, poiché il loro numero è, per ogni
scenario, talmente esiguo, che per avere un batch di dimensione 10 sarebbero
necessari almeno 3 milioni di job in tutto. Tuttavia, con un numero di
$500000$ job totali, si ottengono delle statistiche globali molto attendibili,
infatti, ad esempio per il tempo medio di risposta del sistema, si ottiene 
una dimensione del batch pari a $\lfloor\frac{500000}{64}\rfloor = 7812$.
%
%
\subsection{Scenario 1: $\mathbf{S=N=20}$}
\subsubsection{Tempi di risposta}
\begin{table}[!h]
\begin{tabular}{c|r@{.}l|r@{.}l|r@{.}l}
& \multicolumn{2}{|c|}{$s_1^{clet}$}
& \multicolumn{2}{|c|}{$s_2^{clet}$}
& \multicolumn{2}{|c}{$s_{clet}$} 
\\          
\hline
R1      & $2$&$2264 \pm 0.0071$ & $2$&$4014 \pm 0.0119$ & $2$&$2919 \pm 0.0049$ \\
R2      & $2$&$2223 \pm 0.0076$ & $2$&$4231 \pm 0.0110$ & $2$&$2984 \pm 0.0049$ \\
R3      & $2$&$2270 \pm 0.0073$ & $2$&$4130 \pm 0.0133$ & $2$&$2971 \pm 0.0046$ \\
R4      & $2$&$2249 \pm 0.0078$ & $2$&$4084 \pm 0.0106$ & $2$&$2940 \pm 0.0047$ \\
R5      & $2$&$2240 \pm 0.0067$ & $2$&$4162 \pm 0.0117$ & $2$&$2973 \pm 0.0055$ \\
R6      & $2$&$2261 \pm 0.0072$ & $2$&$3995 \pm 0.0110$ & $2$&$2917 \pm 0.0051$ \\
R7      & $2$&$2235 \pm 0.0070$ & $2$&$4099 \pm 0.0107$ & $2$&$2940 \pm 0.0049$ \\
R8      & $2$&$2200 \pm 0.0073$ & $2$&$4047 \pm 0.0141$ & $2$&$2894 \pm 0.0056$ \\
R9      & $2$&$2137 \pm 0.0069$ & $2$&$4172 \pm 0.0119$ & $2$&$2906 \pm 0.0048$ \\
R10     & $2$&$2179 \pm 0.0080$ & $2$&$4096 \pm 0.0092$ & $2$&$2905 \pm 0.0050$ \\
STIMA   & $2$&$2222$            & $2$&$3904$            & $2$&$3248$            \\
MAX ERR & $0$&$0121$            & $0$&$0436$            & $0$&$0298$            
\end{tabular}
\centering
\caption{Confronto tempi risposta cloudlet per $S=20$}
\label{clet_srv_20}
\end{table}

\begin{table}[!h]
\begin{tabular}{c|r@{.}l|r@{.}l|r@{.}l}
& \multicolumn{2}{|c|}{$s_1^{cloud}$}
& \multicolumn{2}{|c|}{$s_2^{cloud}$}
& \multicolumn{2}{|c}{$s_{cloud}$} 
\\          
\hline
R1      & $2$&$2264 \pm 0.0071$ & $2$&$4014 \pm 0.0119$ & $2$&$2919 \pm 0.0049$ \\
R2      & $2$&$2223 \pm 0.0076$ & $2$&$4231 \pm 0.0110$ & $2$&$2984 \pm 0.0049$ \\
R3      & $2$&$2270 \pm 0.0073$ & $2$&$4130 \pm 0.0133$ & $2$&$2971 \pm 0.0046$ \\
R4      & $2$&$2249 \pm 0.0078$ & $2$&$4084 \pm 0.0106$ & $2$&$2940 \pm 0.0047$ \\
R5      & $2$&$2240 \pm 0.0067$ & $2$&$4162 \pm 0.0117$ & $2$&$2973 \pm 0.0055$ \\
R6      & $2$&$2261 \pm 0.0072$ & $2$&$3995 \pm 0.0110$ & $2$&$2917 \pm 0.0051$ \\
R7      & $2$&$2235 \pm 0.0070$ & $2$&$4099 \pm 0.0107$ & $2$&$2940 \pm 0.0049$ \\
R8      & $2$&$2200 \pm 0.0073$ & $2$&$4047 \pm 0.0141$ & $2$&$2894 \pm 0.0056$ \\
R9      & $2$&$2137 \pm 0.0069$ & $2$&$4172 \pm 0.0119$ & $2$&$2906 \pm 0.0048$ \\
R10     & $2$&$2179 \pm 0.0080$ & $2$&$4096 \pm 0.0092$ & $2$&$2905 \pm 0.0050$ \\
STIMA   & $2$&$2222$            & $2$&$3904$            & $2$&$3248$            \\
MAX ERR & $0$&$0121$            & $0$&$0436$            & $0$&$0298$            
\end{tabular}
\centering
\caption{Confronto tempi di risposta cloud per $S=20$}
\label{cloud_srv_20}
\end{table}

\begin{table}[!h]
\begin{tabular}{c|r@{.}l|r@{.}l|r@{.}l}
& \multicolumn{2}{|c|}{$s_1$}
& \multicolumn{2}{|c|}{$s_2$}
& \multicolumn{2}{|c}{$s$} 
\\          
\hline
R1      & $2$&$2264 \pm 0.0071$ & $2$&$4014 \pm 0.0119$ & $2$&$2919 \pm 0.0049$ \\
R2      & $2$&$2223 \pm 0.0076$ & $2$&$4231 \pm 0.0110$ & $2$&$2984 \pm 0.0049$ \\
R3      & $2$&$2270 \pm 0.0073$ & $2$&$4130 \pm 0.0133$ & $2$&$2971 \pm 0.0046$ \\
R4      & $2$&$2249 \pm 0.0078$ & $2$&$4084 \pm 0.0106$ & $2$&$2940 \pm 0.0047$ \\
R5      & $2$&$2240 \pm 0.0067$ & $2$&$4162 \pm 0.0117$ & $2$&$2973 \pm 0.0055$ \\
R6      & $2$&$2261 \pm 0.0072$ & $2$&$3995 \pm 0.0110$ & $2$&$2917 \pm 0.0051$ \\
R7      & $2$&$2235 \pm 0.0070$ & $2$&$4099 \pm 0.0107$ & $2$&$2940 \pm 0.0049$ \\
R8      & $2$&$2200 \pm 0.0073$ & $2$&$4047 \pm 0.0141$ & $2$&$2894 \pm 0.0056$ \\
R9      & $2$&$2137 \pm 0.0069$ & $2$&$4172 \pm 0.0119$ & $2$&$2906 \pm 0.0048$ \\
R10     & $2$&$2179 \pm 0.0080$ & $2$&$4096 \pm 0.0092$ & $2$&$2905 \pm 0.0050$ \\
STIMA   & $2$&$2222$            & $2$&$3904$            & $2$&$3248$            \\
MAX ERR & $0$&$0121$            & $0$&$0436$            & $0$&$0298$            
\end{tabular}
\centering
\caption{Confronto tempi di risposta sistema per $S=20$}
\label{system_srv_20}
\end{table}

\begin{table}[!h]
\begin{tabular}{c|r@{.}l|r@{.}l}
& \multicolumn{2}{|c|}{$s_{setup}$}
& \multicolumn{2}{|c}{$s_{intr}$}
\\          
\hline
R1      & $2$&$2264 \pm 0.0071$ & $2$&$4014 \pm 0.0119$ \\
R2      & $2$&$2223 \pm 0.0076$ & $2$&$4231 \pm 0.0110$ \\
R3      & $2$&$2270 \pm 0.0073$ & $2$&$4130 \pm 0.0133$ \\
R4      & $2$&$2249 \pm 0.0078$ & $2$&$4084 \pm 0.0106$ \\
R5      & $2$&$2240 \pm 0.0067$ & $2$&$4162 \pm 0.0117$ \\
R6      & $2$&$2261 \pm 0.0072$ & $2$&$3995 \pm 0.0110$ \\
R7      & $2$&$2235 \pm 0.0070$ & $2$&$4099 \pm 0.0107$ \\
R8      & $2$&$2200 \pm 0.0073$ & $2$&$4047 \pm 0.0141$ \\
R9      & $2$&$2137 \pm 0.0069$ & $2$&$4172 \pm 0.0119$ \\
R10     & $2$&$2179 \pm 0.0080$ & $2$&$4096 \pm 0.0092$ \\
STIMA   & $2$&$2222$            & $2$&$3904$            \\
MAX ERR & $0$&$0121$            & $0$&$0436$            
\end{tabular}
\centering
\caption{Confronto tempi di risposta setup e interruzioni per $S=20$}
\label{setintr_srv_20}
\end{table}

\begin{table}[!h]
\begin{adjustbox}{width=\textwidth}
\begin{tabular}{c|c|c|c|c|c|c}
                & R1 & R2 & R3 & R4 & STIMA & MAX ERR \\
                \hline
$n_1^{clet}$  &$8.9260\pm0.0050$& $8.8170\pm0.0130$ & $8.9151\pm0.0082$ &
$8.8811\pm0.0075$ & & \\
$n_2^{clet}$  & R1 & R2 & R3 & R4 & STIMA & MAX ERR \\
$n_{clet}$    & R1 & R2 & R3 & R4 & STIMA & MAX ERR \\
$n_1^{cloud}$ & R1 & R2 & R3 & R4 & STIMA & MAX ERR \\
$n_2^{cloud}$ & R1 & R2 & R3 & R4 & STIMA & MAX ERR \\
$n_{cloud}$   & R1 & R2 & R3 & R4 & STIMA & MAX ERR \\
$n_{setup}$   & R1 & R2 & R3 & R4 & STIMA & MAX ERR \\
$n_1$         & R1 & R2 & R3 & R4 & STIMA & MAX ERR \\
$n_2$         & R1 & R2 & R3 & R4 & STIMA & MAX ERR \\
$n$           & R1 & R2 & R3 & R4 & STIMA & MAX ERR \\
\end{tabular}
\end{adjustbox}
%\centering
\caption{Risultati popolazione media per $S=20$}
\label{pop_20}
\end{table}

\subsubsection{Popolazione}
\subsubsection{Throughput}
%
%
\subsection{Scenario 2: $\mathbf{S=\frac{N}{2}=10}$}
\subsubsection{Tempi di risposta}
\subsubsection{Popolazione}
\subsubsection{Throughput}
%
%
\subsection{Scenario 3: $\mathbf{S=\frac{3}{4}N=15}$}
\subsubsection{Tempi di risposta}
\subsubsection{Popolazione}
\subsubsection{Throughput}
%
%
\subsection{Scenario 4: $\mathbf{S=\frac{N}{4}=5}$}
\subsubsection{Tempi di risposta}
\subsubsection{Popolazione}
\subsubsection{Throughput}
