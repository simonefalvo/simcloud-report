\section{Introduzione}
In questa relazione viene descritta una possibile soluzione ad un problema di
edge computing in cui è richiesto di ottimizzare le prestazioni di un sistema di
computazione cloud, calibrando i parametri di un algoritmo per l'inoltro di task
utente verso un cloudlet ed un server remoto.

Il sistema è in grado di eseguire task di due diverse classi in un cloudlet,
situato ad un ``hop'' di distanza dall'utente, fintanto che le risorse lo
consentono, altrimenti vengono inoltrati ed eseguiti in un server remoto.
Inoltre, se la somma del numero di job presenti nel cloudlet è uguale al valore
di soglia S e se vi è almeno un job di classe 2 in esecuzione, questo viene
fatto migrare dal cloudlet al server remoto per far posto ad un task di classe 1
in arrivo, che ha una maggiore ``priorità'' di esecuzione nel cloudlet.\\
L'obbiettivo principale è quello di trovare il valore ottimale del parametro S
affinché sia minimizzato il tempo di risposta medio.

Il problema è stato affrontato definendo un modello a code per il sistema in
modo tale da poter valutare, tramite l'analisi e la simulazione, i tempi di
risposta e le altre metriche di performance come il throughput e la popolazione
media.\\ 
Il modello è stato validato analiticamente tramite lo studio dello stato del
sistema a regime, calcolandone la distribuzione stazionaria e valutando le
metriche di interesse in relazione ai possibili scenari.\\ 
La simulazione è stata realizzata implementando in linguaggio C un programma
basato su eventi ed i risultati sono stati raccolti ed elaborati utilizzato il
metodo ``batch means'' per avere una stima del comportamento del sistema a
regime.

Nel seguito di questo documento verranno presentati i vari modelli di
astrazione del sistema con i risultati che ne derivano, ed infine verrà fatto un
cofronto in relazione ai possibili scenari analizzati.
