\section{Conclusioni}
Tramite il confronto dei risultati della simulazione nei diversi scenari è stato
possibile valutare le performance del sistema e delle sue componenti al variare
del valore di soglia $S$, e stabilire la bontà del modello analitico mettendone
in evidenza pregi e difetti.

Da tale confronto, sono emerse le seguenti conclusioni relative al modello
analitico:
\begin{itemize}
\item il modello è particolarmente affidabile nella stime delle metriche
relative ai job di classe 1, con una percentuale di errore massimo inferiore
all'$1\%$;
\item nel calcolo della percentuale dei job interrotti di classe 2, si commette
un errore significativo che si ripercuote, in maniera più o meno accentuata, sul
resto delle metriche che coinvolgono le interruzioni dei job, tuttavia tale
errore si attenua nelle metriche globali ed in quelle relative al sistema,
consentendo un stima comunque soddisfacente della visione complessiva del
sistema;
\item l'affidabilità della stima di una metrica locale, dipende strettamente
dalla quantità dei job che vengono processati nel nodo di interesse, per esempio
è stato impossibile ottenere una stima accettabile riguardo ai job di classe 1
che sono stati eseguiti nel cloud e si è riscontrato un errore massimo non
trascurabile nel caso di metriche relative a job di classe 2 eseguiti nel
cloudlet per un valore di soglia pari a 5.
\end{itemize}

In merito alla simulazione, in particolare alle differenze nel comportamento del
sistema in relazione al valore del parametro $S$, è emerso che i fattori che
incidono maggiormente sono legati, come è intuibile, alle interruzioni ed agli
inoltri diretti al cloud, poiché vi corrispondono i tempi di risposta maggiori.

In conclusione si è ottenuto che per un valore di $S=20$ si ottiene il
tempo di risposta minimo, pari a circa $3.61$ secondi, tuttavia, in tale caso,
la stima commette un errore massimo del $2.7\%$. 

Per un valore di soglia $S=5$, il tempo di risposta ($3.65$ secondi) non è molto
lontano dal valore minimo che assume nel caso in cui $S=20$, però la stima è
molto più affidabile, infatti si commette sempre un errore massimo inferiore
all'$1\%$.  Pertanto, se si ha l'esigenza di avere un sistema altamente
predicibile, con tale modello analitico risulterebbe vantaggioso configurare il
sistema con un valore di soglia pari a 5.

Qualora si volesse replicare gli esperimenti effettuati, il codice della
simulazione è disponibile al seguente indirizzo:
\url{https://github.com/smvfal/simcloud}. Ogni simulazione è stata eseguita con
un seme iniziale pari a $12345$, su una macchina con sistema operativo
GNU/Linux.
Per eseguire l'esperimento è necessario eseguire prima il programma
\emph{cloudq} che produce i file con i valori osservati e li salva nella
cartella \emph{data}, infine eseguire i programmi con il prefisso \emph{bm\_}
per il calcolo degli intervalli di confidenza.
